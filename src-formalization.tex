\subsection{Producer-Biased Calculus (PBC)}

\subsubsection{Syntax}

The syntax is defined in \cref{fig:PBC:syntax}.

\begin{figure}[H]
    \setlength{\abovedisplayskip}{0pt}
    \setlength{\belowdisplayskip}{0pt}
    \setlength{\abovedisplayshortskip}{0pt}
    \setlength{\belowdisplayshortskip}{0pt}
  \[
  \begin{array}{lclr}
    \multicolumn{4}{r}{
      \mathcal{C} \in \textsc{Prd}
      \quad
      \mathcal{D} \in \textsc{Con}
      \quad
      T \in \textsc{Type}
      \mspace{60mu}
      \emph{Names}
    }
    \\
    \multicolumn{4}{r}{
      x \in \textsc{Var}
      \mspace{60mu}
      \emph{Variables}
    }
    \\[0.5cm]

    % Contexts
    \Gamma
    & \Coloneqq
    & \varnothing\ |\ \Gamma, x : T
    & \emph{Context}
    \\[0.5cm]

    % Terms
    t
    & \Coloneqq
    & x
    & \emph{Variable}
    \\
    & | & \mathcal{C}\overline{t}
    & \emph{Producer Application}
    \\
    & | & t.\mathcal{D}\overline{t}
    & \emph{Consumer Application}
    \\[0.5cm]

    % Consumer Functions
    cfun
    & \Coloneqq
    & \define\
      T.\mathcal{D}\Gamma : T
      := \match\
         \overline{\mathcal{C}\Gamma \Rightarrow t}
    & \emph{Consumer Function}
    \\[0.5cm]

    % Producer Functions
    pfun
    & \Coloneqq
    & \define\
      \mathcal{C}\Gamma : T
      := \comatch\
         \overline{\mathcal{D}\Gamma \Rightarrow t}
    & \emph{Producer Function}
    \\[0.5cm]

    % Type Declarations
    D
    & \Coloneqq
    & \data\
      T\
      \where\
      \overline{\mathcal{C}\Gamma : T}\
      \with\
      \overline{cfun}
    & \emph{Data Declaration}
    \\
    & | & \codata\
          T\
          \where\
          \overline{T.\mathcal{D}\Gamma : T}\
          \with\
          \overline{pfun}
    & \emph{Codata Declaration}
    \\[0.5cm]

    % Programs
    P
    & \Coloneqq
    & \overline{D}
    & \emph{Program}
    \\[0.5cm]
  \end{array}
  \]
  \caption{Syntax PBC.}
  \label{fig:PBC:syntax}
\end{figure}

\subsubsection{Terms}

\begin{prooftree}
  \AxiomC{}
  \RightLabel{\textsc{Tm-Var}}
  \UnaryInfC{$\typetmpbc{\Gamma, x : T}{x}{T}$}
\end{prooftree}
\vspace*{0.3cm}

\begin{prooftree}
  \AxiomC{$
    \mathcal{C}\Gamma_{\mathcal{C}}
    \in
    \textsc{Prd}(T)
  $}
  \AxiomC{$
    \forall i,\
    \typetmpbc
      {\Gamma}
      {t_{i}}
      {\Gamma_{\mathcal{C}}(i)}
  $}
  \RightLabel{\textsc{Tm-Producer-App}}
  \BinaryInfC{$
    \typetmpbc
      {\Gamma}
      {\mathcal{C}\overline{t_{i}}}
      {T}
  $}
\end{prooftree}
\vspace*{0.3cm}

\begin{prooftree}
  \AxiomC{$
    (\mathcal{D}\Gamma_{\mathcal{D}}, T^{\prime})
    \in
    \textsc{Con}(T)
  $}
  \AxiomC{$
    \forall i,\
    \typetmpbc
      {\Gamma}
      {t_{i}}
      {\Gamma_{\mathcal{D}}(i)}
  $}
  \AxiomC{$\Gamma \vdash t : T$}
  \RightLabel{\textsc{Tm-Consumer-App}}
  \TrinaryInfC{$
    \typetmpbc
      {\Gamma}
      {t.\mathcal{D}\overline{t_{i}}}
      {T^{\prime}}
  $}
\end{prooftree}
\vspace*{0.3cm}

\subsubsection{Well-formedness}

\begin{prooftree}
  \AxiomC{$
    \forall i,\
    \typetmpbc{\Gamma, \Gamma_{i}}{t_{i}}{T^{\prime}}
  $}
  \RightLabel{\textsc{Wf-CFun}}
  \UnaryInfC{$
    \typewf{
      \define\
      T.\mathcal{D}\Gamma : T^{\prime}
      := \match\
         \overline{
           \mathcal{C}_{i}\Gamma_{i}
           \Rightarrow
           t_{i}
         }
    }
  $}
\end{prooftree}
\vspace*{0.3cm}

\begin{prooftree}
  \AxiomC{$
    \forall i,\
    \typetmpbc{\Gamma, \Gamma_{i}}{t_{i}}{T_{i}}
  $}
  \RightLabel{\textsc{Wf-PFun}}
  \UnaryInfC{$
    \typewf{
      \define\
      \mathcal{C}\Gamma : T
      := \comatch\
         \overline{
           \mathcal{D}_{i}\Gamma_{i}
           \Rightarrow
           t_{i}
         }
    }
  $}
\end{prooftree}
\vspace*{0.3cm}

\begin{prooftree}
  \AxiomC{$\typewf{\overline{cfun}}$}
  \RightLabel{\textsc{Wf-Data}}
  \UnaryInfC{$
    \typewf{
      \data\
      T\
      \where\
      \overline{\mathcal{C}_{i}\Gamma_{i} : T}\
      \with\
      \overline{cfun}
    }
  $}
\end{prooftree}
\vspace*{0.3cm}

\begin{prooftree}
  \AxiomC{$\typewf{\overline{pfun}}$}
  \RightLabel{\textsc{Wf-Codata}}
  \UnaryInfC{$
    \typewf{
      \codata\
      T\
      \where\
      \overline{T.\mathcal{D}_{i}\Gamma_{i} : T_{i}}\
      \with\
      \overline{pfun}
    }
  $}
\end{prooftree}
\vspace*{0.3cm}

\begin{minipage}{0.4\textwidth}
  \begin{prooftree}
    \AxiomC{}
    \RightLabel{\textsc{Wf-Prog-Nil}}
    \UnaryInfC{$\typewf{()}$}
  \end{prooftree}
\end{minipage}
\begin{minipage}{0.4\textwidth}
  \begin{prooftree}
    \AxiomC{$\typewf{P}$}
    \AxiomC{$\typewf{D}$}
    \RightLabel{\textsc{Wf-Prog-Cons}}
    \BinaryInfC{$\typewf{(P,D)}$}
  \end{prooftree}
\end{minipage}

\subsubsection{Operational Semantics}

\[
  \begin{array}{lclr} 
    % Values
    v
    & \Coloneqq
    & \mathcal{C}\overline{v}
    & \emph{Value}
    \\[0.5cm]

    % Evaluation Contexts
    ctx
    & \Coloneqq
    & \square
    \\
    & | & \mathcal{C}(\overline{v},ctx,\overline{t})
    \\
    & | & ctx.\mathcal{D}\overline{t}
    \\
    & | & v.\mathcal{D}(\overline{v},ctx,\overline{t})
    & \emph{Evaluation Context}
    \\[0.5cm]
  \end{array}
\]

\begin{prooftree}
  \AxiomC{$
    \mathcal{C}\Delta
    \Rightarrow
    t
    \in
    \textsc{Cases}(\mathcal{D})
  $}
  \RightLabel{\textsc{E-Match}}
  \UnaryInfC{$
    \mathcal{C}\overline{v}.\mathcal{D}\overline{v}^{\prime}
    \evaluatesto
    t[\overline{v}][\overline{v}^{\prime}]
  $}
\end{prooftree}
\vspace*{0.3cm}

\begin{prooftree}
  \AxiomC{$
    \mathcal{D}\Delta
    \Rightarrow
    t
    \in
    \textsc{Cocases}(\mathcal{C})
  $}
  \RightLabel{\textsc{E-Comatch}}
  \UnaryInfC{$
    \mathcal{C}\overline{v}.\mathcal{D}\overline{v}^{\prime}
    \evaluatesto
    t[\overline{v}][\overline{v}^{\prime}]
  $}
\end{prooftree}
\vspace*{0.3cm}

\begin{prooftree}
  \AxiomC{$t \evaluatesto t^{\prime}$}
  \RightLabel{\textsc{E-Cong}}
  \UnaryInfC{$ctx[t] \evaluatesto ctx[t^{\prime}]$}
\end{prooftree}

\subsubsection{Type Safety}

\begin{theorem}[Preservation]
  If $\typetmpbc{}{t}{T}$, and $t \evaluatesto t^{\prime}$, then $\typetmpbc{}{t^{\prime}}{T}$.
\end{theorem}

\begin{theorem}[Progress]
  If $\typetmpbc{}{t}{T}$, then either $t$ is a value or $t \evaluatesto t^{\prime}$ for some $t^{\prime}$.
\end{theorem}

\subsection{Symmetric Calculus (SC)}

In the formalization of the language we use an orientation parameter $\pol \in \{ \polprd,\polcon \}$ and functions

\begin{align*}
  \widehat{\polprd}
  :=\ \polcon
  \quad
  \widehat{\polcon}
  :=\ \polprd
\end{align*}

\subsubsection{Syntax}

The syntax is defined in \cref{fig:SC:syntax}.

\begin{figure}[H]
    \setlength{\abovedisplayskip}{0pt}
    \setlength{\belowdisplayskip}{0pt}
    \setlength{\abovedisplayshortskip}{0pt}
    \setlength{\belowdisplayshortskip}{0pt}
  \[
  \begin{array}{lclr}
    \multicolumn{4}{r}{
      \mathcal{X} \in \textsc{App}
      \quad
      T \in \textsc{Type}
      \mspace{60mu}
      \emph{Names}
    }
    \\
    \multicolumn{4}{r}{
      x \in \textsc{Var}
      \mspace{60mu}
      \emph{Variables}
    }
    \\[0.5cm]

    % Contexts
    \Gamma
    & \Coloneqq
    & \varnothing\ |\ \Gamma, x \prd T\ |\ \Gamma, x \con T
    & \emph{Context}
    \\[0.5cm]

    % Terms
    t
    & \Coloneqq
    & x\ |\ \mathcal{X}\overline{t}
    & \emph{Term}
    \\[0.5cm]

    % Commands
    c
    & \Coloneqq
    & t \mkCmd t\ |\ \Done
    & \emph{Command}
    \\[0.5cm]

    % o-Functions
    fun^{\pol}
    & \Coloneqq
    & \define\
      \mathcal{X}\Gamma \prdcon T
      := \match\
         \overline{\mathcal{X}\Gamma \Rightarrow c}
    & \emph{$\pol$-Function}
    \\[0.5cm]

    % o-Type Declarations
    D^{\pol}
    & \Coloneqq
    & \pol\
      \type\
      T\
      \where\
      \overline{\mathcal{X}\Gamma}\
      \with\
      \overline{fun^{\flip{\pol}}}
    & \emph{$\pol$-Type Declaration}
    \\[0.5cm]

    % Programs
    P
    & \Coloneqq
    & \varnothing\ |\ P, D^{\polprd}\ |\ P, D^{\polcon}
    & \emph{Program}
    \\[0.5cm]
  \end{array}
  \]
  \caption{Syntax SC.}
  \label{fig:SC:syntax}
\end{figure}

\subsubsection{Terms}

\begin{prooftree}
  \AxiomC{}
  \RightLabel{\textsc{Tm-$\pol$-Var}}
  \UnaryInfC{$\typetm{\Gamma, x \prdcon T}{x}{T}$}
\end{prooftree}
\vspace*{0.3cm}

\begin{prooftree}
  \AxiomC{$
    \mathcal{X}\Gamma_{\mathcal{X}}
    \in
    \pol\textsc{-App}(T)
  $}
  \AxiomC{$
    \forall i,\
    \textbf{if}
    \,
    \polprd = \textsc{Orient}(\Gamma_{\mathcal{X}}(i))
    \,
    \textbf{then}
    \,
    \typetmprd
      {\Gamma}
      {t_{i}}
      {\Gamma_{\mathcal{X}}(i)}
    \,
    \textbf{else}
    \,
    \typetmcon
      {\Gamma}
      {t_{i}}
      {\Gamma_{\mathcal{X}}(i)}
  $}
  \RightLabel{\textsc{Tm-$\pol$-App}}
  \BinaryInfC{$
    \typetm
      {\Gamma}
      {\mathcal{X}\overline{t_{i}}}
      {T}
  $}
\end{prooftree}
\vspace*{0.3cm}

\subsubsection{Commands}

\begin{prooftree}
  \AxiomC{$\typetmprd{\Gamma}{t_{1}}{T}$}
  \AxiomC{$\typetmcon{\Gamma}{t_{2}}{T}$}
  \RightLabel{\textsc{Cmd-Cut}}
  \BinaryInfC{$\typecmd{\Gamma}{t_{1} \mkCmd t_{2}}$}
\end{prooftree}
\vspace*{0.3cm}

\begin{prooftree}
  \AxiomC{}
  \RightLabel{\textsc{Cmd-Done}}
  \UnaryInfC{$\typecmd{\Gamma}{\Done}$}
\end{prooftree}

\subsubsection{Well-formedness}

\begin{prooftree}
  \AxiomC{$
    \forall i,\
    \typecmd{\Gamma, \Gamma_{i}}{c_{i}}
  $}
  \RightLabel{\textsc{Wf-$\pol$-Fun}}
  \UnaryInfC{$
    \typewf{
      \define\
      \mathcal{X}\Gamma \prdcon T
      := \match\
         \overline{
           \mathcal{X}_{i}\Gamma_{i}
           \Rightarrow
           c_{i}
         }
    }
  $}
\end{prooftree}
\vspace*{0.3cm}

\begin{prooftree}
  \AxiomC{$\typewf{\overline{fun^{\flip{\pol}}}}$}
  \RightLabel{\textsc{Wf-$\pol$-type}}
  \UnaryInfC{$
    \typewf{
      \pol\
      \type\
      T\
      \where\
      \overline{\mathcal{X}_{i}\Gamma_{i}}\
      \with\
      \overline{fun^{\flip{\pol}}}
    }
  $}
\end{prooftree}
\vspace*{0.3cm}

\begin{minipage}{0.4\textwidth}
  \begin{prooftree}
    \AxiomC{}
    \RightLabel{\textsc{Wf-Prog-Nil}}
    \UnaryInfC{$\typewf{()}$}
  \end{prooftree}
\end{minipage}
\begin{minipage}{0.4\textwidth}
  \begin{prooftree}
    \AxiomC{$\typewf{P}$}
    \AxiomC{$\typewf{D^{\pol}}$}
    \RightLabel{\textsc{Wf-$\pol$-Prog-Cons}}
    \BinaryInfC{$\typewf{(P,D^{\pol})}$}
  \end{prooftree}
\end{minipage}

\subsubsection{Operational Semantics}
\begin{prooftree}
  \AxiomC{$
    \mathcal{X}_{i}\Gamma_{i}
    \Rightarrow
    c_{i}
    \in
    \polprd\textsc{-Cases}(\mathcal{X})
  $}
  \RightLabel{\textsc{E-$\polprd$-Cut}}
  \UnaryInfC{$
    \left(
      \mathcal{X}
      \overline{t}
      \mkCmd
      \mathcal{X}_{i}
      \overline{t_{i}}
    \right)
    \evaluatesto_{\cmd}
    c_{i}
    [\overline{t}]
    [\overline{t_{i}}]
  $}
\end{prooftree}
\vspace*{0.3cm}

\begin{prooftree}
  \AxiomC{$
    \mathcal{X}_{i}\Gamma_{i}
    \Rightarrow
    c_{i}
    \in
    \polcon\textsc{-Cases}(\mathcal{X})
  $}
  \RightLabel{\textsc{E-$\polcon$-Cut}}
  \UnaryInfC{$
    \left(
      \mathcal{X}_{i}
      \overline{t_{i}}
      \mkCmd
      \mathcal{X}
      \overline{t}
    \right)
    \evaluatesto_{\cmd}
    c_{i}
    [\overline{t}]
    [\overline{t_{i}}]
  $}
\end{prooftree}

\subsubsection{Type Safety}

\begin{theorem}[Preservation]
  If $\typecmd{}{c}$, and $c \evaluatesto_{\cmd} c^{\prime}$, then $\typecmd{}{c^{\prime}}$.
\end{theorem}

\begin{theorem}[Progress]
  If $\typecmd{}{c}$, then either $c$ is $\Done$ or $c \evaluatesto_{\cmd} c^{\prime}$ for some $c^{\prime}$.
\end{theorem}

\subsection{Translation from PBC to SC}

$\mathfrak{T}$ for types ... and $\mathfrak{S}$ for terms ...

\subsubsection{Properties}

\begin{lemma}[...]
  If ...
\end{lemma}

\begin{theorem}[...]
  If ...
\end{theorem}

\subsection{Termination Analysis}

TODO:

\begin{enumerate}
  \item Formalize type-graph stuff
  \item Formalize call-graph stuff
  \item Formulate termination analysis
  \item Prove termination analysis sound
\end{enumerate}

\subsection{TEMPORARY-SECTION: Second-Order Lambda-Calculus With Recursion (RLC2)}

This section is the legacy from a previous but unsuccessful attempt using RLC2.
It is only still around because together with the Agda code in the repo it could help contributors to understand the termination issues of this project better.

\begin{figure}[H]
    \setlength{\abovedisplayskip}{0pt}
    \setlength{\belowdisplayskip}{0pt}
    \setlength{\abovedisplayshortskip}{0pt}
    \setlength{\belowdisplayshortskip}{0pt}
  \[
  \begin{array}{lclr}
    \multicolumn{4}{r}{
      x \in \textsc{Term}
      \quad
      X \in \textsc{Type}
      \mspace{60mu}
      \emph{Variables}
    }
    \\[0.5cm]

    % Contexts
    \Gamma
    & \Coloneqq
    & \varnothing\ |\ \Gamma, x : T
    & \emph{Context}
    \\[0.5cm]

    % Terms
    t
    & \Coloneqq
    & x\ |\ \mathcal{X}\overline{t}
    & \emph{Term}
    \\[0.5cm]

    % Constants
    c
    & \Coloneqq
    & \text{in}\ |\ \text{out}\ |\ \text{in$^{-1}$}\ |\ \text{out$^{-1}$}
    & \emph{Constant}
    \\[0.5cm]

    % Terms
    s,t
    & \Coloneqq
    & x
    & \emph{Term Variable}
    \\
    & | & \lambda x : T.t
    & \emph{Term Abstraction}
    \\
    & | & t \, t
    & \emph{Term Application}
    \\
    & | & \Lambda X.t
    & \emph{Type Abstraction}
    \\
    & | & t \, T
    & \emph{Type Application}
    \\
    & | & \rho x : T.t
    & \emph{?}
    \\
    & | & c(X,T)
    & \emph{?}
    \\[0.5cm]

    % Types
    S,T
    & \Coloneqq
    & X
    & \emph{Type Variable}
    \\
    & | & T \to T
    & \emph{Function Type}
    \\
    & | & \Pi X.T
    & \emph{Universal Type}
    \\
    & | & \mu X.T
    & \emph{Recursive Type}
    \\
    & | & \nu X.T
    & \emph{Corecursive Type}
    \\[0.5cm]
  \end{array}
  \]
  \caption{Syntax RLC2}
  \label{fig:RLC2:syntax}
\end{figure}

To ensure that recursive and corecursive types really exist we introduce a simple kinding as follows

\vspace*{0.3cm}
\begin{minipage}{0.15\textwidth}
  \begin{prooftree}
    \AxiomC{}
    \RightLabel{\textsc{$\ast$-Var}}
    \UnaryInfC{$X : \ast$}
  \end{prooftree}
\end{minipage}
\begin{minipage}{0.4\textwidth}
  \begin{prooftree}
    \AxiomC{$T_{1} : \ast$}
    \AxiomC{$T_{2} : \ast$}
    \RightLabel{\textsc{$\ast$-Fun}}
    \BinaryInfC{$T_{1} \to T_{2} : \ast$}
  \end{prooftree}
\end{minipage}
\begin{minipage}{0.4\textwidth}
  \begin{prooftree}
    \AxiomC{$T : \ast$}
    \RightLabel{\textsc{$\ast$-Uni}}
    \UnaryInfC{$\Pi X.T : \ast$}
  \end{prooftree}
\end{minipage}

\vspace*{0.3cm}
\begin{minipage}{0.4\textwidth}
  \begin{prooftree}
    \AxiomC{$T : \ast$}
    \AxiomC{$\text{Pos}(X,T)$}
    \RightLabel{\textsc{$\ast$-Rec}}
    \BinaryInfC{$\mu X.T : \ast$}
  \end{prooftree}
\end{minipage}
\begin{minipage}{0.4\textwidth}
  \begin{prooftree} 
    \AxiomC{$T : \ast$}
    \AxiomC{$\text{Pos}(X,T)$}
    \RightLabel{\textsc{$\ast$-Corec}}
    \BinaryInfC{$\nu X.T : \ast$}
  \end{prooftree}
\end{minipage}
\vspace*{0.3cm}

Here $\text{Pos}(X,T)$ means that $X$ is contained only strictly positive in $T$.
Strictly positive types $T$ w.r.t. $X$ are those which do not contain $X$ in the left subtree of any $\to$.

The typing rules for System F are standard:
The rules for the simply typed lambda calculus

\vspace*{0.3cm}
\begin{minipage}{0.2\textwidth}
  \begin{prooftree}
    \AxiomC{$x : T \in \Gamma$}
    \RightLabel{\textsc{Tm-Var}}
    \UnaryInfC{$\Gamma \vdash x : T$}
  \end{prooftree}
\end{minipage}
\begin{minipage}{0.375\textwidth}
  \begin{prooftree}
    \AxiomC{$\Gamma,x : T_{1} \vdash t : T_{2}$}
    \RightLabel{\textsc{Tm-Abs}}
    \UnaryInfC{$\Gamma \vdash \lambda x : T_{1}.t : T_{1} \to T_{2}$}
  \end{prooftree}
\end{minipage}
\begin{minipage}{0.4\textwidth}
  \begin{prooftree}
    \AxiomC{$\Gamma \vdash t : T_{1} \to T_{2}$}
    \AxiomC{$\Gamma \vdash s : T_{1}$}
    \RightLabel{\textsc{Tm-App}}
    \BinaryInfC{$\Gamma \vdash t \, s : T_{2}$}
  \end{prooftree}
\end{minipage}

\vspace*{0.3cm}
together with the rules for abstraction of types and application to types

\vspace*{0.3cm}
\begin{minipage}{0.4\textwidth}
  \begin{prooftree}
    \AxiomC{$\Gamma,X \vdash t : T$}
    \RightLabel{\textsc{Ty-Abs}}
    \UnaryInfC{$\Gamma \vdash \Lambda X.t : \Pi X.T$}
  \end{prooftree}
\end{minipage}
\begin{minipage}{0.4\textwidth}
  \begin{prooftree}
    \AxiomC{$\Gamma \vdash t : \Pi X.T$}
    \AxiomC{$S : \ast$}
    \RightLabel{\textsc{Ty-App}}
    \BinaryInfC{$\Gamma \vdash t \, S : T [S]$}
  \end{prooftree}
\end{minipage}

\vspace*{0.3cm}
The rule for general recursion:

\vspace*{0.3cm}
\begin{prooftree}
  \AxiomC{$\Gamma, x : S \to T \vdash t : S \to T$}
  \RightLabel{\textsc{$\rho$-Abs}}
  \UnaryInfC{$\rho x : S \to T. t : S \to T$}
\end{prooftree}

\vspace*{0.3cm}
The rules for $\mu$-abstractions $\mu X.T$ and $\nu$-abstrations $\nu Y.T$ are:

\vspace*{0.3cm}
\begin{minipage}{0.45\textwidth}
  \begin{prooftree}
    \AxiomC{$\mu X.T : \ast$}
    \RightLabel{\textsc{$\mu$-Intro}}
    \UnaryInfC{$\text{in}(X,T) : T[\mu X.T] \to \mu X.T$}
  \end{prooftree}
\end{minipage}
\begin{minipage}{0.45\textwidth}
  \begin{prooftree}
    \AxiomC{$\nu X.T : \ast$}
    \RightLabel{\textsc{$\nu$-Intro}}
    \UnaryInfC{$\text{out}(X,T) : \nu X.T \to T[\nu X.T]$}
  \end{prooftree}
\end{minipage}

\vspace*{0.3cm}
\begin{minipage}{0.45\textwidth}
  \begin{prooftree}
    \AxiomC{$\mu X.T : \ast$}
    \RightLabel{\textsc{$\mu$-Intro'}}
    \UnaryInfC{$\text{in}^{-1}(X,T) : \mu X.T \to T[\mu X.T]$}
  \end{prooftree}
\end{minipage}
\begin{minipage}{0.45\textwidth}
  \begin{prooftree}
    \AxiomC{$\nu X.T : \ast$}
    \RightLabel{\textsc{$\nu$-Intro'}}
    \UnaryInfC{$\text{out}^{-1}(X,T) : T[\nu X.T] \to \nu X.T$}
  \end{prooftree}
\end{minipage}

\vspace*{0.3cm}
For the sake of simplicity we sugar the language with binary products and sums

\begin{minipage}{0.45\textwidth}
  \begin{align*}
    S \times T
    &:= \Pi X. (S \to T \to X) \to X
    \\
    (s,t)
    &:= \Lambda X. \lambda (f : S \to T \to X). f \, s \, t
    \\
    \text{pr}_{1}
    &:= \lambda p : S \times T. p \, S \, (\lambda x. \lambda y. x)
    \\
    \text{pr}_{2}
    &:= \lambda p : S \times T. p \, T \, (\lambda x. \lambda y. y)
  \end{align*}
\end{minipage}
\begin{minipage}{0.45\textwidth}
  \begin{align*}
    S + T
    &:= \Pi X. (S \to X) \to (T \to X) \to X
    \\
    \text{in}_{1}
    &:= \lambda x : S. \Lambda X. \lambda f : S \to X. \lambda g :T \to X. f \, x
    \\
    \text{in}_{2}
    &:= \lambda x : T. \Lambda X. \lambda f : S \to X. \lambda g :T \to X. g \, x
    \\
    f + g
    &:= \lambda c : S + T. c \, T' \, f \, g
  \end{align*}
\end{minipage}

\vspace*{0.3cm}
This easily generalizes to $n$-ary products and sums.

\subsubsection{Mendler-Style}

\vspace*{0.3cm}
What makes an inductive type \enquote{Mendler-Style} is replacing $T[Y] \to Y$ by $\Pi X.(X \to Y) \to T \to Y$ in the ordinary recursion principle which is justified since these types are isomorphic by Yoneda.
Of course, there are dual considerations for coinductive types.
So:

\vspace*{0.3cm}
\begin{prooftree}
  \AxiomC{$\mu X.T : \ast$}
  \RightLabel{\textsc{$\mu$-Rec}}
  \UnaryInfC{$\text{rec}(X,T) : \Pi Y.(\Pi X.(X \to Y) \to T \to Y) \to \mu X.T \to Y$}
\end{prooftree}

\vspace*{0.3cm}
\begin{prooftree}
  \AxiomC{$\nu X.T : \ast$}
  \RightLabel{\textsc{$\nu$-Rec}}
  \UnaryInfC{$\text{corec}(X,T) : \Pi Y.(\Pi X.(Y \to X) \to Y \to T) \to Y \to \nu X.T$}
\end{prooftree}

Reduction:

\vspace*{0.3cm}
\begin{minipage}{0.25\textwidth}
  \begin{enumerate}
    \item[(1)] $(\lambda x : T.t) s \evaluatesto t[s]$
    \item[(2)] $(\lambda x : T.t \, x) \evaluatesto t$
  \end{enumerate}
\end{minipage}
\begin{minipage}{0.25\textwidth}
  \begin{enumerate}
    \item[(3)] $(\Lambda X.t) S \evaluatesto t[S]$
    \item[(4)] $(\lambda X.t \, X) \evaluatesto t$
  \end{enumerate}
\end{minipage}
\begin{minipage}{0.45\textwidth}
  \begin{enumerate}
    \item[(5)] $\text{rec} \, S \, s \, (\text{in} \, t) \evaluatesto s \, \mu X.T \, (\text{rec} \, S \, s) \, t$
    \item[(6)] $\text{out} \, (\text{corec} \, S \, s \, \, t) \evaluatesto s \, \nu X.T \, (\text{corec} \, S \, s) \, t$
  \end{enumerate}
\end{minipage}
